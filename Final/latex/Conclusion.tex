\chapter*{Conclusion et Comparaison entre les deux Filtres EKF-UKF}
\addcontentsline{toc}{chapter}{Conclusion}
Notre étude démontre clairement l'impact des types de mesures sur l'estimation de l'état d'un système dynamique. Lorsque seules les positions sont mesurées, l'incertitude des vitesses augmente, ce qui peut conduire à des erreurs dans l'estimation globale. 
En revanche, la disponibilité des mesures de vitesse améliore considérablement la précision des estimations, permettant une convergence plus rapide du filtre. 
Enfin, lorsque seules les vitesses sont accessibles, la précision des positions diminue, soulignant la nécessité d'avoir un ensemble de mesures diversifié pour obtenir des estimations fiables. 


Cette analyse souligne l'importance d'un choix judicieux des mesures dans le cadre de l'application des filtres de Kalman pour la localisation précise d'une cible en mouvement.